\chapter{La communication chimique}
\section{Les phéromones}
\subsection{Les Substances sémiochimiques}

Les substances sémiochimiques sont des molécules organiques synthétisées
par un organisme vivant et qui interviennent comme moyen de
communication que ce soit de manière intraspécifique ou de manière
interspécifique. Ces substances sémiochimiques modifie le comportement
ou la physiologie du récepteur.

TODO Image Les actions interspécifiques, ont lieu grâce aux substances
allélochimiques, ce sont des actions concernant les relations entre
différentes espèces alors que les actions intraspécifiques, ayant lieu
grâce aux phéromones sont relatif aux rapports qui se produisent au sein
d'une même espèce. C'est pour cette raison, que nous allons nous
intéresser aux phéromones pour étudier la communication chimique qui a
lieu au sein d'une même espèce, les fourmis.

\subsubsection{Définition de phéromone}

La notion de phéromone a été introduite par Karlson et Lüsher en 1959,
ils en ont donné la définition suivante : « Une phéromone est une
substance (ou un mélange de substances) qui, après avoir été sécrétée à
l'extérieur par un individu (émetteur), est perçue par un individu de la
même espèce (récepteur) chez lequel elle provoque une réaction
comportementale spécifique, voire une modification physiologique. »Le
mot phéromone vient du grec pherein « transporter » et hormân «exciter
». Les phéromones sont produites par des glandes spécifiques et sont
secrétées à l'extérieur d'un organisme. Comme les enzymes, elles
agissent dans des quantités minimes. Nous allons étudier un peu plus
tard ces glandes spécifiques dans la partie émission.

Généralement ; on ne parle pas d'une seule phéromone mais plutôt d'un
bouquet phéromonale. En effet ; les phéromones ne sont pas des corps
purs mais des mélanges de différentes substances chimiques. Le message
transmis est spécifique aux autres fourmis, ainsi sa composition de
point de vue qualitatif mais aussi quantitatif prend tout son sens.

\subsubsection{Les différents types de phéromones :}

Le célèbre scientifique Wilson, en 1962, distingue les phéromones de
déclanchement aux phéromones modificatrices selon leurs modes d'action.
Selon lui : - « Les phéromones de déclanchement produisent un changement
d'état immédiat et réversible dans le comportement du récepteur ». Il y
classe les phéromones sexuelles (attractives ou aphrodisiaques), les
phéromones d'alarme, les phéromones de piste, les phéromones
d'agrégation, les phéromones de territoire\ldots{}

\begin{itemize}
\itemsep1pt\parskip0pt\parsep0pt
	\item « Les phéromones modificatrices élaborent une suite de modifications physiologiques chez le récepteur, sans aucun changement immédiat dans son comportement ». cependant, ces modifications physiologiques lui
	permettent ultérieurement d'acquérir un nouveau répertoire
	comportemental, pouvant se manifester suite à une sii tuation donnée. De
	plus ; ces phéromones réagissent dans le déterminisme des castes chez
	les insectes sociaux comme les fourmis ou les abeilles. En effet ;
	Wilson décrit que la reine peut créer une « substance royale » qui
	développent les ovaires des ouvrières, les empêchant ainsi de la
	construction à l'intérieur de la ruche de substances royales.
\end{itemize}

Dans ces deux catégories ; on distingue différents types de phéromones,
nous allons donc nous intéresser aux différentes phéromones de
déclanchement car ce sont les phéromones les plus étudiées par les
scientifiques.

\begin{itemize}
\item
  phéromones de territoire : utilisé pour marquer le territoire et pour
  le repérer, il s'agit de phéromones utilisées par certains mammifères
  comme les chats ou les chiens mais aussi par les poissons. On les
  retrouve dans les urines. En ce qui concerne les fourmis ; les
  phéromones territoriales assure la sécurité de la fourmilière. Ainsi ;
  elles sont déposées pour marquer l'abord du nid. Elles sont secrétées
  par la glande de Dufour. Elle exerce deux rôles principaux.
  Premièrement ; elle permet aux ouvrières de la fourmilière de se
  repérer pour rejoindre leur nid. Et enfin, elle porte une action
  répulsive envers les fourmis d'autres colonies.
  
\item
  phéromones d'alarme : comme leur noms l'indique, ces phéromones
  utilisées par les insectes, les poissons, et mammifères permettent aux
  animaux d'alerter d'un danger. Chez les fourmis, on constate que la
  réaction agit de façon collective. En effet, dès lors qu'une fourmi
  sécrète une phéromone d'alarme, les autres fourmis présentent
  réagissent et sécrètent à leurs tours cette même phéromone ainsi la
  réaction devenant collective permet de mieux contrôler les dangers. On
  en conclut alors que le rôle premier de ces substances chimiques est
  d'avertir les autres congénères de la fourmilière d'un danger. Selon
  BLUM et PASSERA, « ces phéromones constituent un progrès dans
  l'évolution des espèces eusociales ». Ce phénomène peut être observé
  avec les jets à plus centimètre d'acide fourmis donc voici un extrait
  montrant de « c'est pas sorcière » qui illustre cette situation. (TO
  DO: mettre vidéo)
\end{itemize}

Le comportement d'alarme peut-être divisé en deux activités. Tout
d'abord, on constate « un mouvement de panique », ainsi les différentes
fourmis adoptent un comportement agressif auquel s'ajoute un déplacement
à grande vitesse vers le danger. Puis, les fourmis adoptent une attitude
d'attaque, elle ouvre leurs mandibules et attaque l'insecte dangereux en
piquant pour déposer du venin.

Comme nous allons l'abord plus tard, il faut noter que les phéromones
d'alarme sont secrétées par les glandes mandibulaire, la glande de
Dufour, ou encore la glande à poison.

Généralement, ces substances chimiques forment des chaînes carbonées
plus courtes que celle des phéromones d'alarmes comme nous le montre ces
exemples dans les tableaux suivants. C'est pourquoi, elles sont plus
volatiles et ainsi elles possèdent une durée de vie courte. Ces
phéromones d'alarmes sont souvent des cétones (comme l'octan-2-one que
nous avons synthétisé voir partie LB), des esters, des alcools, des
hydrocarbures comme l'exemple du décane.

Différentes phéromones d'alarmes sont utilisées par les mêmes espèces.
Elles sont donc moins spécifiques que les phéromones de piste, ainsi
l'acide formique est non seulement utilisé par les formicas rufa
(fourmis des bois) mais aussi par les Camponotus. Chez certaines espèces
ont pu tracer différents cercles démontrant les champs d'action des
phéromones. Par exemple, la fourmi tisserande d'Afrique Oecophylla
longinoda secrète grâce à ces glandes mandibulaires 4 substances
principales qui agissent dans un périmètre particulier. En voici, le
schéma : (TODO mettre le schéma)

L'hexanal, composé organique de la famille des aldéhydes, est la
substance chimique la plus volatile ; elle déclenche un état d'alerte
chez les fourmis se trouvant à une dizaine de centimètre de la source
qui a été mordue. Lorsque les fourmis ouvrières s'approchent et se
trouvent à cinq centimètres, elles sont attirées par le centre mais le
1-hexanol, substance chimique qualifiable de « lourde » agit comme
répulsif cependant son temps de vie étant limité. Cette action répulsive
permet d'attirer plus de fourmis dans le cercle afin de permettre à un
ensemble de fourmis de s'attaquer collectivement et efficacement. Peu à
peu, l'effet répulsif diminue par évaporation. Ainsi, le 3-undécanone et
le 2-butyl-2-oclénal se trouvant au centre du dispositif sont les
substances chimiques les plus « lourdes », permettent de déclencher un
comportement de morsure. ( To do,tableaux 6-Dimethyl-5-Heptan-1-Al )

\begin{itemize}
\itemsep1pt\parskip0pt\parsep0pt

	\item
	Les phéromones de piste : Utilisées par les insectes sociaux ainsi que
	les mammifères, ces phéromones permettent de tracer une piste
	chimique. Elles sont à l'origine d'un recrutement d'autres fourmis.
	Elles sont utilisées pour l'approvisionnement de nourriture à la
	fourmilière mais aussi pour neutraliser un ennemi ou pour le
	déménagement à un nouveau nid. Pour réaliser cette piste chimique, la
	fourmi secrète la phéromone en frottant son aiguillon ou l'extrémité
	de son abdomen sur le sol. Cette méthode permet l'augmentation
	progressive du nombre de fourmi recruté.
  
\end{itemize}

Les chercheurs ont démontré qu'il existe au moins 9 glandes permettant
la sécrétion de phéromones de piste. La plus part de ces phéromones se
trouvent dans l'abdomen par exemple dans la glande de Dufour ou dans les
glandes à poison. Nous pouvons réaliser une expérience permettant de
démontrer l'utilisation d'une piste chimique : Prenons une fourmi qui va
de la fourmilière vers une source de nourriture. La première fourmi va
secréter une phéromone qui « traversera son chemin » de retour à sa
fourmilière. Les autres fourmis sont alertées et suivront la même piste
qu'a suivie la première fourmi pour apporter de la nourriture à la
fourmilière.

(to do: mettre Schéma des parcours utilisé lors des expériences et
observation de différents moments)

On constate que plus les fourmis font des aller- retour, plus elles ont
tendance à emprunter le chemin le plus court. Ainsi, le bouquet
phéromone des phéromones de piste permet aux fourmis de marquer un «
trajet » qu'elle signale à d'autres fourmis. De plus ; les substances
secrétées permettent aussi aux autre fourmis de connaître la quantité de
nourriture disponible qui à leur tour sécrèteront ce bouquet jusqu'à
temps que la nourriture est disponible. Exemple de phéromones de piste
que nous avons modélisées grâce au logiciel ChemSketch ; il faut noter
que ce logiciel ne prend pas en compte les liaisons doubles: (to do
mettre le tableau)

On doit ajouter à cela que les scientifiques Regnier et Wilson ont
calculé l'efficacité de la perception des phéromones d'alarme chez
certaines espèces. Ainsi, les Acanthomyops claviger perçoivent les
phéromones d'alarme dès lors que la concentration atteint 10E10 à 10E12
molécules/cm3. Et chez les Atta Texana, l'efficacité est atteinte dès
lors que la concentration est de 10E8 molécules/cm3. ( voir pour mettre
les puissances ac Valentin)

\begin{itemize}
\item
  LLes phéromones d'agrégation : Les phéromones d'agrégation sont un
  véritable outil chez les fourmis pour assurer la cohésion sociale.
  Ainsi, elles ont un rôle important pour la reproduction, pour
  l'hibernation, pour nidifier, pour estiver ou encore la protection
  social. Ces phéromones ont une durée de vie très variable les unes des
  autres. En effet ; elles peuvent agir temporairement ou de façon
  permanente pour assurer la cohésion sociale. Voici un exemple de
  phéromones d'agrégation, il s'agit d'une phéromone spécifique des
  Coléoptères. En effet ; il existe très peu d'étude concernant les
  phéromones d'agrégation des fourmis. (to do mettre le tableau)
\item
  Les phéromones passeports : elles sont comparables à une « carte
  d'identité » de la fourmi. Ainsi, elles sont imprégnées sur leurs
  cuticule et permettent aux autres fourmis de les identifier par
  rapport à leur fourmilière à partir d'un contact antennaire ces
  phéromones sont certes des substances chimiques mais leurs modes de
  communication est tactile et nom chimique c'est pour cela que nous
  n'allons pas les détailler. Cependant ; on peut aussi considérer que
  le nid imprégner de phéromones passeport permet aux colonies de se
  reconnaître les uns par aux autres. Voici un exemple de phéromones
  passeport :(to do mettre le tableau)
\item
  Les phéromones sexuelles : Elément essentiel lors de la reproduction,
  les phéromones sexuelles permettent aux fourmis d'attirer les fourmis
  du sexe opposé. Il faut noter que ces phéromones sont la plupart du
  temps émise par les femelles afin d'attirer les mâles. Selon des
  études, la plupart des phéromones émise par les femelles sont des
  hydrocarbure. La longueur de la chaine carbonique moyenne est de 10 à
  20 atomes de carbone. Elles permettent aussi de connaître le sexe de
  l'individu. Ces phéromones sont de véritables signaux olfactifs
  présents chez de nombreux insectes.
\end{itemize}

On distingue deux catégories de phéromones sexuelles : les substances
d'appels secrétés par des glandes en dehors de l'appareil génital ainsi
que les substances aphrodisiaques qui entrainent l'accouplent. C'est le
cas de certains alcaloïdes Voici un exemple de phéromone sexuelle qui
est une cétone : (TODO mettre tableau)

\begin{itemize}
\itemsep1pt\parskip0pt\parsep0pt
\item
  Les phéromones de recrutement : elles permettent de recruter d'autres
  fourmis afin de donner une aide. Elle provoque un regroupement de
  plusieurs fourmis en un point précis. Elles sont utilisées par exemple
  pour l'approvisionnement en nourriture. Elles sont un enchaînement des
  phéromones de piste. Ces phéromones sont aussi utilisées lors d'un
  déménagement d'un nid à l'autre. Il faut noter que les phéromones
  d'alarme provoque aussi un regroupement mais celui-ci est défensif, de
  plus ce n'est pas le rôle premier des phéromones d'alarme. C'est pour
  cela qu'il ne faut pas les confondre avec les phéromones de
  recrutement.
\end{itemize}

(mettre le tableau)

\textbf{Bilan sur les différents types de phéromone utilisé dans la
communication chimique des fourmis :} Tous les phéromones que nous avons
cités précédemment sont des molécules organique car elles possèdent
principalement les éléments carbone et hydrogène. Généralement ; la
longueur de la chaine carbonée témoigne de l'efficacité de la molécule.
Certaines phéromones sont spécifique de l'espèce (c'est le cas de la
phéromone d'agrégation des coléoptères) mais elles sont dans d'autre cas
utilisé par plusieurs insectes. D'autre part; certaines phéromones sont
utilisés dans le même but comme par exemple la phéromone de piste et les
phéromones de recrutement lors de l'approvisionnement en nourriture.
Ainsi, le message émis par les fourmis ne résulte pas d'une seule
phéromone mais de différents types de phéromones constituant le bouquet
phéromonale.

\subsection{Emission des phéromones}

\subsubsection{Synthèse et le transport des phéromones}

Les fourmis produisent des messagers chimiques par le biais de cellules
sécrétrices dont l'ensemble forme des glandes. Ces messages chimiques se
répondent dans l'environnement. En effet~; il s'agit de substance
chimique volatiles~: des phéromones qui possèdent une durée de vie
limité. Il faut noter que l'effet de phéromone résulte de la taille de
la molécule. En outre~; on constate que plus la durée d'action de la
molécule est longue plus la molécule est grande. Ces substances
chimiques sont synthétisées par une fourmi émettrice afin de transmettre
une information à une fourmi réceptrice. Ainsi, ces phéromones peuvent
être transmise par le transport de l'air, le placement au sol, et par
contact.

\subsubsection{La structure des glandes exocrine et les différents types de glandes.}
\paragraph{ Structure des glandes exocrines}

Définition~: Une glande exocrine est un organe sécrétant des substances
chimique, dans notre cas il s'agit des phéromones, qui sont rejetée à
l'extérieur de l'organisme des fourmis. De plus, ces glandes ont aussi
pour rôle de synthétisé les phéromones qu'elles sécrètent. Il existe au
moins trois types de glande exocrine pouvant être distingué selon le
tableau suivant~: (to do mettre tableau) Nous pouvons voir la structure
de ces glandes grâce au schéma ci-dessous et ainsi étudier l'évacuation
des phéromones.

(to do mettre tableau)

L'évacuation des phéromones dépend des différents pores de la glande.
Selon le type de cellules où se trouve la phéromone, l'évacuation est
différente. Pour les cellules de types 1~: cellules glandulaire
épidermique, la phéromone est rejetée par canaux très étroit qui se
situe dans la cuticule. En ce qui concerne les cellules de type 2~: les
cellules glandulaires intra-épidermique, la phéromone est évacué par par
des canaux fin ou par les chambres d'évacuation suite à leurs transport
par les cellules adjacents. Enfin, les cellules de type 3~: les cellules
sous-épidermique, la phéromone est rejetée par un réservoir central
grâce à un canal cuticulaire. \#\#\#\# 1.1.1. Les différentes glandes
exocrines (to do mettre le schéma)

\begin{itemize}
\item
  Glande mandibulaire~: Située en dessous des mandibules, la glande
  mandibulaire synthétise les phéromones d'alarme. Elles produisent
  aussi un liquide peu fluide qui permet de malaxer et de ramollir la
  nourriture retrouvé par les fourmis. Ces substances chimiques sont
  principalement constituées de citronellol et de citronellal qui ont
  respectivement une fonction défensif et déclencheur d'alarme. D'autre
  part~; ces glandes mandibulaire permettent du maintien de la
  hiérarchie social
\item
  Glande métathoracique : elle produise des substances chimiques qui
  contribuent sans doute au repérage ou à la défense.
\item
  Glande Anale : elle contient principalement des esters volatiles. Ces
  glandes produisent l'acide formique qui est une substance chimique
  défensive. Cet acide peut être projeté à plusieurs centimètres.
\item
  Les glandes pygidiale : elle produit des phéromones d'alarme et permet
  ainsi d'alerter les autres fourmis. La glande de Dufour ou alcaline :
  surnommée le «~flacon à parfum~» des fourmis. Il s'agit d'une
  phéromone de piste, qui est le moyen de recruter les congénères, grâce
  au traçage d'une piste chimique. Cette substance chimique est active
  durant 100 secondes. Ces phéromones sont ensuite rejetées par
  l'aiguillon.
\item
  La glande à poison ou à venin~: comme la glande anale, elle produit de
  l'acide formique, il s'agit donc d'une glande qui synthétise des
  phéromones d'alarme qu'elle utilise face à un danger. Cette glande,
  chez certaine espèce, rejette du venin qui a un rôle paralysant, d'où
  son nom. Ces substances chimiques sont ensuite secrétées par
  l'aiguillon.
\end{itemize}

\subsection{Réception d'un signal chimique : les antennes, des récepteurs sensoriels}

Chez la fourmi comme tous autre insectes, la réception des signaux
chimique ce fait par le biais de récepteurs sensoriel. Ces
chimiorécepteur sont de cellules sensorielles spécialisés qui se situe
au niveau des organes olfactif c'est-à-dire les antennes. Les fourmis
possèdent toutes deux antennes constituée de cellules sensorielles nous
pouvons les observer grâce à un microscope optique à balayage. (to do
mettre photo)

Nous observons ainsi que les antennes sont constitués de cils. En effet,
il s'agit de sensilles. Chaque sensille réagit à une phéromone ou un
bouquet phéromones. Lorsque nous observons de plus près ces sensilles,
on remarque qu'ils sont constitués de plusieurs pores qui permettent la
traversée des phéromones. (mettre schema to do)

Les sensilles sont constituées de dendrites. Il s'agit de neurones
sensoriels qui permettent la réception et la transmission d'un signal
chimique auparavant qui se transforme en signal électrique aux corps
nerveux composés de cellules sensorielles. Il faut noter qu'une sensille
peut contenir de 2 à 5 neurones olfactifs. Les cellules sensorielles
sont reliées à l'axone qui permet de transférée le message électrique au
cerveau des fourmis qui l'analysera pour obtenir des concernant une
source alimentaire retrouvée par une fourmi (la qualité, ou sa
quantité). (mettre le schéma )

On trouve, au niveau des dendrites, des récepteurs olfactif permettant
l'identification du stimulus (éléments extérieurs capables de déclencher
des phénomènes spéciaux dans l'organisme : odeur, phéromones). Les
récepteurs olfactifs se composent d'une protéine et d'un glucide qui
forme un site actif auquel se fixe la molécule (la ou les phéromone(s)).
Ainsi les récepteurs olfactifs sont spécifiques d'une molécule ou
plusieurs molécules (pour les bouquets phéromonaux). En effet ; les
phéromones sont transportées dans le liquide sensillaire grâce aux OBP
(olfactory binding proteins) et les PBP (pheromone binding protein), qui
sont des protéines spécifiques.

\section{Utilisation de ces connaissances par l'homme}

\subsection{Synthèse d'une phéromone par l'homme}

Afin de se servir de ces connaissance en matière de communication
chimique, l'homme doit copier le système de communication chimique
c'est-à-dire les phéromones, en les synthétisant. On doit étudier
précisément la composition phéromonale des glandes, afin de synthétiser
une phéromone laboratoire comme par exemple l'octan-2-one. Les synthèses
des phéromones permettent à l'homme de les utilisée en tant qu'outil
contre la destruction culturale comme nous allons le voir un peu plus
tard. Nous avons donc procéder à une expérience afin de synthétiser de
l'octan-2-one, une phéromone qui est une cétone grâce à l'octan-2-ol qui
est un alcool. Cette phéromone se trouve dans les glandes mandibulaires
des fourmis. Voici les représentations du réactif principal et du
produit. (to do mettre le tableau)

Nous avons donc suivi un protocole expérimentale réalisé par un
enseignent de l'académie de Nantes que nous avons modifié afin de
répondre au mieux aux contraintes de matériel de laboratoire.

\emph{1. Mise en place des réactifs et déroulement de la transformation
chimique} \emph{Introduire dans le ballon le turbulent et, à l'aide
d'éprouvettes graduées, sous la hotte :} \emph{- 5 mL d'octan-2-ol,}
\emph{- 10 mL d'acide éthanoïque.} \emph{Mettre l'agitateur magnétique
en fonctionnement.} \emph{Suivre la diminution de la température jusqu'à
15°C avec un thermomètre, avant de placer l'ensemble bouchon + tulipe +
tube.} \emph{Introduire dans l'ampoule de coulée isobare 40mL d'eau de
javel à 24 °chl.}

Afin de doser la concentration de l'eau de javel à 24 °chl comme le
recommande le protocole. Nous avons étudié la définition du degré
chlorométrique. Ainsi, Le degré chlorométrique d est le nombre de litres
de dichlore gazeux Cl2, pris dans les conditions normales de température
et de pression, qu'il faut dissoudre dans un litre d'une solution
d'hydroxyde de sodium pour obtenir un litre d'eau de javel titrant d
°chl, selon la réaction : Cl2(g) + 2Na+ + 2OH- 2 Na+ + ClO- + Cl- + H2O
Dans des conditions normal (de pression et de température normal) on
doit dissoudre 24L de dichlore gazeux dans 1L de soude afin d'obtenir 1L
d'eau de Javel (to do mettre photo)

\emph{Lorsque la température est inférieure à 15°C, faire couler l'eau
de javel goutte à goutte en veillant à ne pas dépasser la température de
25°C. L'addition doit être lente (au goutte à goutte). } \emph{Lorsque
l'ampoule de coulée est vide, enlever le cristallisoir et laisser
revenir à la température ambiante tout en agitant pendant 15 minutes.
Enlever l'ampoule de coulée et la rincer.} \emph{La solution doit rester
de couleur jaunâtre ce qui prouve que l'eau de javel est en excès par
rapport à l'octan-2-ol. Si l'eau de javel n'est pas en excès, ajouter
quelques mL d'eau de javel dans le mélange réactionnel.} \emph{En fin de
synthèse, remettre l'ampoule de coulée en place et y placer environ 5mL
d'une solution d'hydrogénosulfite de sodium (Na+ + HSO ). Faire couler
goutte à goutte la solution d'hydrogénosulfite de sodium dans le mélange
réactionnel jusqu'à décoloration totale de la solution}

(to do mettre les photos)

\emph{2. Lavages de la phase organique} \emph{Récupérer le turbulent.
Transvaser le contenu du ballon à l'aide d'un entonnoir dans une ampoule
à décanter, rincer le ballon avec 20mL d'eau distillée froide et
récupérer cette eau de lavage dans l'ampoule à décanter.} \emph{Ajouter
dans l'ampoule à décanter 50mL d'une solution aqueuse de chlorure de
sodium saturée préalablement refroidie dans le bain eau + glace pilée.}
* Agiter l'ampoule à décanter et laisser reposer quelques minutes afin
de séparer la phase aqueuse et la phase organique. Eliminer la phase
aqueuse. La phase organique est peu abondante.\emph{ } Laver la phase
organique avec 50mL d'une solution d'hydrogénocarbonate de sodium à 5
\%, solution préalablement refroidie dans le bain réfrigérant.\emph{ }
Attention au dégagement gazeux !! Lorsque celui-ci est atténué, agiter
l'ampoule à décanter en dégazant l'ampoule plusieurs fois. Laisser
reposer et éliminer la phase aqueuse.\emph{ } Laver une dernière fois la
phase organique avec 50mL d'une solution aqueuse de chlorure de sodium
saturée, préalablement refroidie dans le bain réfrigérant. Eliminer la
phase aqueuse et recueillir la phase organique dans un erlenmeyer
propre. *

(to do mettre photo)

\emph{3. Séchage de la phase organique} * Sécher la phase organique avec
un sel anhydre comme par exemple du sulfate de magnésium anhydre.*

(to do mettre photo)

\emph{4. Filtrer la phase organique avec un dispositif de Büchner }

(to do mettre photo)

\emph{5. Caractérisation du groupe carbonyle de la cétone obtenue } *
Dans un tube à essai, verser environ 2 mL de solution de D.N.P.H. et y
ajouter quelques gouttes d'octan-2-ol. Agiter le tube à essai et
observer. \emph{ } Refaire la même opération avec quelques gouttes du
liquide organique obtenu au cours de cette synthèse. \emph{ }Conclure. *

La synthèse enfin terminé nous pouvons donner, l'équation de la réaction
d'oxydoréduction lors de cette transformation chimique : ClOH + H+(aq) +
2 e- = Cl- + H2O C8H18O = C8H16O + 2 H+(aq) + 2 e- C8H18O + ClOH ¾¾®
C8H16O + H+(aq) + Cl- + H2O

Grâce à un tableau d'avancement ci-dessous nous pouvons calculer le
rendement maximum de la synthèse de l'octan-2-ol.

(to do mettre le tableau)

Quelque calcul afin de remplir le tableau : m (C8H16O) = 5 ? 0,819 =
V(C8H16O) x d((C8H16O) = 4,09 g\\Donc n(C8H16O) = (m (C\_8 H\_16
O))/(M?(C?\_8 H\_16 O)) = = 3,15.10-2 mol n (HClO) = M(HClO) ? V = 1,07
? 40,10-3 = 4,28.10-2 mol.

Il faut noter que le réactif limitant est l'octan-2-ol. On peut conclure
grâce au tableau d'avancement que le rendement maximum est : Mmax= xmax×
M(C8H16O) = 3,15.10-2 ×128= 4,03g

\subsection{Lutte intégrée}

Depuis de nombreuse année ; l'agriculture repose sur l'utilisation de
pesticide afin de lutte contre les insectes ravageur. Pourtant d'autre
solution plus respectueuse de l'environnement existe. Il s'agit de la
lutte intégrée. Elle a été défini au plan européen comme « l'application
rationnelle d'une combinaison de mesures biologiques, biotechnologiques,
chimiques, physiques, culturales ou intéressant la sélection des
végétaux dans laquelle l'emploi de produits chimiques
phytopharmaceutiques est limité au strict nécessaire pour maintenir la
présence des organismes nuisibles en dessous de seuil à partir duquel
apparaissent des dommages ou une perte économiquement inacceptables. »
Nous allons donc nous intéresser aux mesure chimique que l'homme peut
applique : En étudiant les insectes ravageurs l'homme peut en déduire
les phéromones qu'utilises les ravageurs et ainsi réaliser une synthèse
cette phéromones, il peut ainsi réussir à éliminer ou limiter la
population de ravageur dans la culture choisie comme c'est le cas de la
méthode de la confusion sexuelle qui est mise en pratique grâce à des
moyens biotechniques afin d'utilise la synthèse d'une phéromone, il
s'agit des diffuseurs.

Les avantages des phéromones par rapport aux insecticides :
L'utilisation de phéromones est un enjeu majeur pour l'agriculture car
elles présentent plusieurs avantages face à l'utilisation des
insecticides. Tous d'abord l'utilisation de la synthèse des phéromones
permet de ciblé l'action agricole, c'est-à-dire que l'utilisation des
phéromones permet une sélectivité dans le contrôle des nuisibles, ce qui
est impossible avec les insecticide ; en effet, leurs effet se
manifestent même pour mes insecte utiles, qui sont touchée. En effet ;
les phéromones ne détruisent pas l'équilibre biologique car le
traitement ne vise qu'une seule espèce. De faite ; le traitement est
spécifique d'une seules espèce. D'autre part, les phéromones permettent
de préserver la biodiversité mais aussi l'environnement alors que les
insecticides agissent comme des poisons, ils ne sont pas spécifique
d'une espèce. Ils polluent ainsi l'environnement. L'homme, en utilisant
les phéromones, copie ce qu'utilise les insectes pour communiquer, dans
le but de limite les ravage culturales. L'avantage des phéromones réside
donc aussi dans le fait qu'elles soient biodégradables. Les quantités
nécessaires pour attirer un insecte sont d'environ 15g alors que
l'utilisation des insecticides implique l'utilisation de quantités
importantes. Dès lors, leurs couts s'élève. Donc ; selon tous ces points
il est plus avantageux d'utiliser des phéromones pour palier au mieux à
la préservation de l'environnement.

Méthode de la confusion sexuelle:\\Afin de réduire la population cible
(souvent des insectes), l'homme a recourt à plusieurs méthode comme
celle de la confusion sexuelle. Il utilise ainsi les connaissances qu'il
a accumulées concernant la communication chimique. Ainsi, il sait que
les phéromones sexuelles sont à l'origine de la reproduction. L'homme
utilise la méthode de la confusion sexuelle en libérant des phéromones
sexuelle synthétise en laboratoire. Il désoriente ainsi l'insecte en
agissant sur les chimiorécepteurs de l'animale. Cette libération de
phéromone engendre une fatigue sensorielle ; ainsi, le mâle n'est plus
capable de trouver les femelles. En effet, l'homme a « camoufler » les
phéromones naturelles émises par les femelles. Il existe ainsi une
confusion entre les phéromones d'origine naturelle et ceux d'origine
synthétique. Dès lors, cette méthode réduit le nombre total
d'accouplement ou entraine un retard. Le taux de fécondité est par
conséquent réduit ce qui provoque une diminution de la population cible
sur l'espace où la phéromone sexuelles de synthèse a été diffuser. Cette
méthode de lutte chimique est généralement utilisée contre
micro-lépidoptères de la famille des tortricidés (tordeuses). Elle
préserve ainsi les cultures tels que les vignoble, les cultures de
pommes, ceux de cerise, de prune de maïs. Le piégeage sélectif

On utilise par exemple la méthode du piégeage sélectif, du ravageur mâle
ou de la femelle mais aussi des deux à la fois. Cette méthode est
souvent utilisée pour un suivi du ravageur permettant de constater ou
non leur présence. Elle permet aussi de donner un estimation sur les
insectes ravageurs. C'est aussi un outil permettant de connaître la
période de reproduction et estimer aisnsi la période d'attaque. Voici
quelque exemple de piège utilisé en agriculture :

Les pièges a entonnoir et le pièges à ailette : une capsule permettant
la diffusion de phéromones y est fixe. Chaque capsule est spécifique
d'une seule espèce. Les ravageurs souvent des lépidoptères sont attirés
par la phéromone. Ils volent ainsi tous autour jusqu'à un épuisement
totale pour enfin tomber dans l'entonnoir. Ces pièges sont souvent
destinés à des cultures de petites taille ou pour des jardins. Récipient
contient de l'eau afin que les insectes se noient.