\chapter{La communication ultrasonore}

\section{Introduction}

Certaines gammes d'ultrasons peuvent être entendues et utilisées pour
communiquer par de nombreux animaux vertébrés terrestres, comme les
chiens, certains rongeurs, par des chauves-souris ou des dauphins. Ces
fréquences peuvent s'étendre de 15 kHz jusqu'à 200 kHz selon les
espèces, alors que celles ci sont audibles par l'homme dans un domaine
s'étendant de 20Hz à 20kH. Les ultrasons dans le monde animal sont
étudiés par la bioacoustique. Elle étudie la production, la réception et
l'interprétation des sons chez les animaux et humains. La bioacoustique
étudie l'interprétation des sonorités et l'anatomie des corps. Ainsi
elle a prouvé que de nombreux animaux utilisent des sons émis non
audibles par l'homme. Ces animaux utilisent les ultrasons sous forme de
sonar et les utilisent pour se repérer, se nourrir et communiquer. De
même, c'est grâce à ces fréquences que certains crétacés et
chauves-souris peuvent émettre et réceptionner les ultrasons, ce qui
entraine l'exploitation du système d'écholocation par ces espèces.

\subsection{Point de vue historique}

Il y a encore plusieurs dizaines d'année, la production et la perception
d'ultrasons étaient considérés possibles uniquement chez les mammifères.
En 1973, le scientifique Konishi remarque que les oiseaux n'entendraient
pas le son dont la fréquence dépasse 12 kHz , et selon les données
disponibles dans les années 1980, l'audition des amphibiens étaient
limités à 5 kHz, selon Fay en 1988. Cependant des chercheurs ont ensuite
constaté que des amphibiens anoures et un oiseau sont capables de les
percevoir. Ce sont la grenouille Odorrana tormota et un passereau
chanteur Abroscopus albogularis, ils vivent près de torrents bruyants,
et insèrent dans leur chant des harmoniques d'ultrasons. Ainsi, cette
grenouille est capable d'émettre et de percevoir des ultrasons, de plus
de 100 kHz. C'est la première espèce non mammifère dotée de cette
propriété à avoir été découverte. Le mâle pousse des cris semblables à
un chant d'oiseau et possède une anatomie de l'oreille inhabituelle,
avec notamment un tympan concave, qui contient un canal auditif et des
osselets très légers, donc très sensibles.

\section{L'écholocation}

L'écholocalisation consiste à envoyer des sons et à écouter leur écho
pour localiser et identifier les éléments d'un environnement. Certaines
espèces peuvent émettre des ultrasons comme les chauves-souris. Elles
émettent des ultrasons qui se répercutent sur les objets environnants,
ce qui leur permet ainsi de percevoir leur environnement. Ainsi la
chauve-souris se sert de l'écholocation pour se repérer dans l'obscurité
et pour repérer sa proie. Elle émet des impulsions à haute fréquence. Le
son récupéré sur la trajectoire de la chauve-souris revient vers
celle-ci. L'écholocation permet à la chauve-souris de déterminer la
proximité de l'insecte. Grâce à la réponse continue des impulsions
réfléchies, la chauve-souris se dirige automatiquement vers la proie
pour la capturer. D'ailleurs, les souriceaux perçoivent des ultrasons
émis par leur mère allaitante qui les allaite.

L'écholocalisation a trois principaux inconvénients.

D'une part, elle nécessite la réception et le traitement d'échos très
faibles ; cela implique un appareil auditif très perfectionné (d'où les
oreilles très grandes des chauves-souris) et un appareil neurologique de
traitement très fin. Pour qu'il puisse entendre un écho, il est
nécessaire à l'émetteur de produire des cris très forts, d'où des
fréquences élevées. D'autre part, la cible peut percevoir les sons émis
pour la repérer et réagir en s'échappant. Enfin, chez les animaux vivant
en groupe (chauve-souris et dauphins notamment), l'émission et la
réception de leurs signaux peuvent être brouillés par ceux de leurs
congénères.

\subsection{L'exemple des dauphins}

L'écholocation des dauphins est la capacité de ces deniers à repérer et
situer les aspects importants de leur environnement, leurs congénères ou
les proies. Elle se base sur la propagation des ondes acoustiques dans
l'eau. Le dauphin est capable d'émettre différents types de son, de
fréquences variables, certains servant à communiquer, d'autres à se
repérer dans l'espace. Le système d'émission chez le dauphin est bien
plus complexe que chez l'homme. L'homme n'est en effet capable de
produire que du son audible, c'est-à-dire entre 20 et 20 000 Hz.

\{\% include image.html img=``img/ultrasons/image1.jpeg''
caption=``Schéma des fréquences des ultrasons''\%\}

On peut séparer les ondes acoustiques émises par les dauphins en deux
grands groupes : les sifflements, utilisés pour communiquer, et les
clics servant à l'écholocation{[}1{]}. Le langage des dauphins est mal
connu et semble complexe. Tout au plus peut on dire qu'il présente de
grandes variations en fonction du groupe de dauphins étudié et des
individus. Les sifflements sont bien localisés en fréquence et se
situent plutôt dans les ultrasons (\textless{} 25 kHz). Les sifflements
sont des sons caractéristiques d'une espèce ou d'un individu. Ce sont
des faisceaux couvrant une bande étroite de basse fréquence, et utilisés
pour la communication entre espèces.

Les clics d'écholocation sont des signaux très brefs (quelques dizaines
de microsecondes) et donc répartis sur une large bande spectrale. Par
exemple la largeur de bande à 3 dB renvoie à une cinquantaine de kHz.
Plus le dauphin se rapproche de sa proie, plus le train de clics est
rapide. La résolution maximale que peut traiter le cerveau du dauphin
est de 600 clics par secondes. Ces clics peuvent être représentés selon
deux types. Le premier est sous la forme d'un faisceau large de basse
fréquence, d'émission lente et de longue portée. Ils sont utilisés pour
créer des images de son environnement. Le second type des clics
d'écholocation est représenté par un faisceau étroit à courte portée,
occupant une large bande de haute fréquence. Ils servent à analyser une
potentielle proie ou un objet rencontré.

\{\% include image.html img=``img/ultrasons/image2.png''
caption=``Représentaion temporelle d'un clic 1''\%\}

\{\% include image.html img=``img/ultrasons/image3.png''
caption=``Représentation fréquentielle d'un clic 1''\%\}

\subsection{L'effet Doppler}\label{leffet-doppler}

Pour recevoir les signaux réfléchis par les cibles, le dauphin exploite
des tissus adipeux situés sous sa mâchoire, qui remontent jusqu'à son
oreille interne. Le son est donc transmis à l'oreille interne, puis au
cerveau, qui l'analyse. Le dauphin peut alors déterminer la distance de
la cible, sa taille, ainsi que sa vitesse en mesurant la différence de
fréquence en exploitant l'effet Doppler. C'est à dire le décalage de
fréquence d'une onde entre la mesure à l'émission et la mesure à la
réception lorsque la distance entre l'émetteur et le récepteur varie au
cours du temps. On réserve le terme d'« effet Doppler-Fizeau » aux ondes
électromagnétiques. Le dauphin peut aussi sonder sous les sédiments,
étant donné que le son se propage sous le sable.

L'effet Doppler est le décalage de fréquence d'une onde acoustique ou
électromagnétique entre la mesure à l'émission et la mesure à la
réception lorsque la distance entre l'émetteur et le récepteur varie au
cours du temps elle renseigne aussi sur la vitesse de la cible par
rapport à l'émetteur.

\{\% include image.html img=``img/ultrasons/image4.png''\%\}

La perception d'un signal dépend de la vitesse relative entre la source
et le récepteur Si un observateur en mouvement cherche à mesurer cette
durée, il lui trouve une valeur différente, plus élevée s'il se déplace
dans le sens de l'onde, plus courte s'il se déplace en sens contraire.

\{\% include image.html img=``img/ultrasons/image5.jpeg''\%\}

\{\% include image.html img=``img/ultrasons/image6.jpeg''\%\}

Si la vitesse \(V_r\) est petite devant \[c\]. En termes de longueur
d'onde, \[\lambda = c \times T\], l'observateur perçoit un rayonnement
de longueur d'onde
\[\lambda\prime = \lambda \times (1 + \frac{V_r}{c})\] soit décalé de la
quantité \[\Delta\lambda = \lambda\prime - \lambda\] telle que :
()
\[\Delta\lambda = \frac{V_r}{c}\]

Ce calcul montre que le décalage relatif de la longueur d'onde est
proportionnel au rapport de la vitesse de la source par rapport à
l'observateur à la vitesse de la lumière.

\subsection{L'émetteur}\label{luxe9metteur}

Le dauphin ne possède pas de cordes vocales mais trois pairs de sacs
aériens, de forme différente, situés sous l'évent, de part et d'autre du
conduit nasal. Ces sacs servent de réserves d'air. Le dauphin arrive à
produire des sons en faisant passer l'air d'un sac à l'autre par des
ouvertures dont il règle le diamètre, grâce à un ensemble de muscles et
de nerf. Pour pouvoir orienter de manière précise ses faisceaux sonores
le dauphin est doté d'un organe, le melon. Les ondes sonores créent vont
se réfléchir sur la paroi osseuse frontale concave du crâne de l'animal
pour ensuite se concentrer sur une masse graisseuse qui est le melon. Le
dauphin peut ainsi orienter l'émission de ses ondes ultrasonores où il
le souhaite.

Le larynx est aussi utilisé pour l'émission sonore. Le dauphin peut en
effet expulser de l'air par ses poumons et cet air fait vibrer des
muscles puissants du conduit respiratoire. Ses muscles vont alors
transmettre ses vibrations à des os dont ceux des maxillaires. La
vibration est ensuite transmise à l'extrémité du rostre de façon
unidirectionnelle.

\subsection{Le récepteur}\label{le-ruxe9cepteur}

Le système de réception du dauphin est complexe. Il repose sur deux
récepteurs qui sont les oreilles et le maxillaire inférieur Les sons
sont réceptionnés au niveau de l'orifice auditif externe. Ils passent
par le conduit auditif et arrivent dans l'oreille interne. Ils sont
ensuite transmis au cerveau par l'influx nerveux d'un nerf acoustique.
Les centres acoustiques du cerveau dont le but est d'analyser les
messages sonores sont très développés chez le dauphin. Tout le système
auditif de cet animal est protégé par des masses de mucus qui arrêtent
les vibrations parasites qui pourrait provenir de l'environnent.

Les ondes acoustiques peuvent aussi être reçues par le maxillaire
inférieur, les sons s'y propagent mieux. Les ondes sonores sont donc
reçues par l'extrémité du bec de l'animal, elles se propagent dans un
corps graisseux au niveau du maxillaire et sont transmises à l'oreille
interne au niveau de l'articulation de la mâchoire. Ensuite, tout ce
passe comme ce qui a été expliqué précédemment.

\{\% include image.html img=``img/ultrasons/image7.png''\%\}

\{\% include image.html img=``img/ultrasons/image8.jpeg''\%\}

\subsection{La prestine}\label{la-prestine}

Chez plusieurs espèces douées de compétence d'écholocation par émission
et réception d'ultrasons, dans l'eau ou dans l'air ; certains cétacés et
chauve-souris ont recours à la prestine, protéine impliquée dans
l'écholocation. Des chercheurs se sont concentrés sur cette protéine
présente chez tous les mammifères. Elle est située sur les cellules
cillées externes de la cochlée, dans l'oreille interne, cette protéine
joue un rôle d'amplificateur des ondes sonores (en particulier des
hautes fréquences comme les ultrasons), grâce aux vibrations qu'elle
provoque. Cette protéine est similaire chez les différents groupes de
chauves-souris qui utilisent l'écholocation.

L'équipe sino-britannique de Stephen Rossiter (University of London, GB)
et l'équipe sino-américaine de Jianzhi Zhang (University of Michigan,
E-U) ont comparé la prestine des chauves-souris et celle de différents
cétacés.

En construisant des arbres phylogénétiques uniquement basés sur
l'évolution de la prestine (et non sur les autres caractéristiques de
ces mammifères), les deux équipes ont abouti à des rapprochements
surprenants. Chauves-souris et dauphins doués pour l'écholocation
devenaient cousins, formant un groupe évolutif cohérent.

Les dauphins (en bleu) au milieu des chauve-souris (en noir) et donc
éloignées des baleines dépourvues d'écholocation (en bas en vert).

Ces chercheurs avaient analysés la suite de nucléotides composant le
gène de la prestine chez cinq espèces d'odontocètes, le cachalot et
quatre espèces de dauphin, ainsi que chez des baleines incapables
d'écholocalisation. Puis ils avaient comparé les séquences obtenues à
celles de 18 espèces de chauves-souris. Ils avaient alors mis en
évidence une ressemblance entre les séquences des odontocètes à sonar,
le cachalot mis à part, et celles des chauves-souris.

Donc cela signifie que la version de la protéine codée par le gène
\emph{Prestin} chez les dauphins est plus proche de la version de la
protéine chez les chauve-souris douées d'écholocalisation que de la
version de la protéine chez les baleines dépourvues d'écholocalisation.
Mais comment est-ce possible ?

Au cours de l'évolution, des mutations se sont accumulées dans l'ADN du
gène \emph{Prestin}. Lorsqu'une mutation survient dans un gène, elle
modifie le codon en question, ce qui peut avoir pour conséquence de
changer l'acide aminé. Si l'acide aminé n'est pas modifié, il n'y a pas
de conséquence sur le phénotype. Par contre, si l'acide aminé est
modifié, la protéine a de grandes chances de fonctionner différemment.
Si elle fonctionne mieux, l'organisme se reproduira en moyenne plus que
les autres organismes. Grâce à l'action de la sélection naturelle, une
mutation positive peut être de plus en plus présente dans la population
au cours des générations.

Dans le cas des dauphins et des chauves-souris, les chercheurs cités
ci-dessus ont montré que, par la sélection naturelle, les différences
entre leurs séquences d'ADN du gène \emph{Prestin} correspondent à des
changements vers les mêmes acides aminés. A l'inverse, les différences
entre les séquences d'ADN des dauphins et des baleines dépourvues
d'écholocalisation correspondent à des changements d'acides aminés
différents.

Cela signifie que la prestine joue un rôle important dans
l'écholocalisation, que celle-ci ait été acquise dans l'eau ou dans
l'air. Les mutations du gène de la prestine doivent avoir été
suffisamment fréquentes pour que les individus dotés par hasard de la
bonne séquence de la protéine, celle qui est associée à la meilleure
sensibilité aux ultrasons, soient sélectionnés par leur environnement.

\section{Utilisation de ces connaissances par l'homme}

\subsection{NMMP}

En 1942, la Suède a eu l'idée d'utiliser des phoques pour faire la
chasse aux sous-marins allemands. Ces phoques dressés, équipés d'une
mine sur leur dos, devaient nager sous les coques, puis le moment venu,
la mine explosait avec le corps de l'animal. Cette nouvelle tactique de
guerre a ouvert la voie aux expériences militaires. L'US Navy utilise
des otaries et des dauphins depuis 1960.

Otarie dréssée pour repérer des objets sous-marins.

Le programme de mammifères marins de la marine américaine, en anglais
U.S. Navy Marine Mammal Program (NMMP), est un programme dirigé par la
Marine américaine qui étudie l'emploi militaire de mammifères marins (le
Grand dauphin qui émet et reçoit en un millième de seconde, des
fréquences variant entre 220 et 250 000 hertz et l'otarie de Californie)
et les entraîne à des tâches tels que la protection de navires et de
ports, le repérage et le dégagement de mines, ainsi que la récupération
d'objets. Le programme est basé à San Diego en Californie où les animaux
sont entraînés. Les animaux du NMMP ont été déployés en zones de combat,
notamment pendant la Guerre du Viêt Nam et la Guerre d'Irak. Cette
dernière a permis aux américains d'ouvrir le golfe Arabique au marché
international après la fin du conflit, en particulier grâce a ces
animaux. La Navy a recours à certaines équipes humains-cétacés. Les
équipes MK 4, 7 et 8 utilisent des dauphins; MK 5 utilise des otaries,
et MK 6 utilise à la fois dauphins et otaries. Ces équipes peuvent être
déployées partout sur le globe en 72 heures, vers des zones de conflits.

soldat dauphin

Trois niveaux de profondeur sont visibles : la zone proche de la surface
(``seasonal thermocline'') dont la température dépend de la saison de
l'année, la zone intermédiaire thermocline (``permanent thermocline'')
puis la zone des eaux froides profondes (``deep-water isothermal
layer''). Passée la valeur de la vitesse minimale (``sound velocity
minimum'' : Vmin), la profondeur a donc une influence positive sur la
célérité. Nous pouvons remarquer que la vitesse de propagation des ondes
sonores dans l'eau n'est pas constante. Elle est maximale dans les eaux
de surface puis décroit dans la zone intermédiaire jusqu'à sa vitesse
minimale pour augmenter de nouveau progressivement dans les eaux
profondes.

\begin{enumerate}
\def\labelenumi{\arabic{enumi}.}
\itemsep1pt\parskip0pt\parsep0pt
\item
  Ying Li, Zhen Liu, Peng Shi, Jianzhi Zhang. The hearing gene Prestin
  unites echolocating bats and whales. Current Biology, 2010;
  DOI:10.1016/j.cub.2009.11.042
\end{enumerate}

\{\% include buttons.html \%\}

\{\% include katex\_render.html \%\}
